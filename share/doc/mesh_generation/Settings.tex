% Seitenränder
\geometry{includeheadfoot}	% Kopf und Fußzeile für geometry teil des body
\geometry{vmargin={2.5cm}}	% oberer und unterer Rand
\geometry{inner={3.5cm}}	% innerer Rand (für zweiseitige Dokumente)
\geometry{outer={2.5cm}}	% äußerer Rand (für zweiseitige Dokumente)

% Abstände bei einem Absatz
% 	Abstand zwischen den Absätzen: halbe Höhe vom kleinen x in der aktuellen Schriftart
\setlength{\parskip}{0.5ex}
% 	Einzug am Anfang eines Absatzes: auf Null setzen
\setlength{\parindent}{0ex}

% Nummerierung und Inhaltsverzeichnis
% Bis zu subsubsections numerieren
\setcounter{secnumdepth}{3}
% Aber nur bis zu subsections ins Inhaltsverzeichnis aufnehmen
\setcounter{tocdepth}{2}
% Equation numbers are <chapter>.<eq. in chapter> (Bsp. 1.1, 1.2, ..., 2.1, 2.2)
\numberwithin{equation}{chapter}
% Bibliography DIN but with english features, change to plaindin for german
%\bibliographystyle{plaindin_custom} 
%\bibliographystyle{siam}
\bibliographystyle{abbrv}
	
% Auch bei Fußnoten soll die Linie die gesamte Textbreite haben
\renewcommand{\footnoterule}{%
  \kern -3pt
  \hrule width \textwidth
  \kern 2.6pt
}

% Kopf und Fußzeilen
\pagestyle{scrheadings}
\automark[section]{chapter} 
\ohead[]{\headmark}
\chead[]{}
\ihead[]{}
\ofoot[\pagemark]{\pagemark}
\cfoot[]{}
\ifoot[]{}
\renewcommand*\chapterpagestyle{scrplain}

\KOMAoptions{cleardoublepage=scrplain}


%%%%%%%%%%%%%%%%%%
%% Typeset Code %%
%%%%%%%%%%%%%%%%%%

% define colors
\definecolor{dkgreen}{rgb}{0,0.6,0}
\definecolor{gray}{rgb}{0.5,0.5,0.5}
\definecolor{darkgray}{rgb}{0.3,0.3,0.3}
\definecolor{mauve}{rgb}{0.58,0,0.82}

% define XML style
% \lstdefinelanguage{XML}
% {
%     morestring=[b]",
%     morestring=[s]{>}{<},
%     morecomment=[s]{<?}{?>},
%     stringstyle=\color{black},
%     identifierstyle=\color{darkblue},
%     keywordstyle=\color{cyan},
%     morekeywords={xmlns,version,type},
%     backgroundcolor=\color{lightgray},
%     numbers=left,
%     numberstyle=\footnotesize\ttfamily\color{gray},
%     numbersep=0.5pt
% }

% define listing style
\lstset{
    frame=single, % other options: none|leftline|topline|bottomline|lines|single|shadowbox 
    language=XML,
    aboveskip=3mm,
    belowskip=3mm,
    showstringspaces=false,
    columns=flexible,
    basicstyle={\small\ttfamily},
    numbers=none,
    %float,
    %$morekeywords=[s]{>}{<},
    numberstyle=\tiny\color{gray},
    keywordstyle=\color{blue},
    commentstyle=\color{dkgreen},
    stringstyle=\color{mauve},
    breaklines=true,
    breakatwhitespace=true,
    tabsize=4
}

% define standalone keywords in text
\newcommand{\att}[1]{\texttt{\color{mauve} #1}}
\renewcommand{\tag}[1]{\texttt{\color{blue} #1}}

% paragraphs that are linked
\newcommand{\refpar}[2]{\par \vspace{0.3cm} \hypertarget{#1}{\textbf{#2}}}

% define a style to typeset a shell command
\newcommand{\shellcmd}[1]{ {\color{darkgray} \vspace{0.2cm} \par \indent \hspace{0.5cm} \texttt{\normalsize > #1} \vspace{0.2cm} \par } }

% external link symbol
\newcommand{\ExternalLink}{%
\tikz[x=1.2ex, y=1.2ex, baseline=-0.05ex]{% 
    \begin{scope}[x=1ex, y=1ex]
        \clip (-0.1,-0.1) 
            --++ (-0, 1.2) 
            --++ (0.6, 0) 
            --++ (0, -0.6) 
            --++ (0.6, 0) 
            --++ (0, -1);
        \path[draw, 
            line width = 0.5, 
            rounded corners=0.5] 
            (0,0) rectangle (1,1);
    \end{scope}
    \path[draw, line width = 0.5] (0.5, 0.5) 
        -- (1, 1);
    \path[draw, line width = 0.5] (0.6, 1) 
        -- (1, 1) -- (1, 0.6);
    }
\!\!\!\!
}